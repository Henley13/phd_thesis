%!TEX root = ./main.tex

% acknowledgments section
\chapter*{Acknowledgement} \addcontentsline{toc}{chapter}{Acknowledgement} \adjustmtc
\markboth{Acknowledgement}{Acknowledgement}

Au moment d'écrire les derniers mots de ce manuscrit, l'occasion se présente d'apprécier le chemin parcouru.
Si l'exercice d'une thèse reste une tâche solitaire, sa réussite n'en demeure pas moins liée aux rencontres que l'on fait.
J'aimerais remercier ici toutes les personnes qui ont pu compter ces dernières années, par leur présence et leur soutien.\\

Mes premiers remerciements vont bien évidemment à Thomas et Florian.
Un très grand merci pour m'avoir fait confiance et avoir accepté d'encadrer ma thèse.
Merci d'avoir fait le chercheur que je suis aujourd'hui, merci pour votre bonne humeur, votre optimisme et votre écoute.
Je trouve que nous avons formé une belle équipe et nos (parfois trop longues) réunions vont me manquer !
J'aimerais également exprimer ma gratitude envers Edouard Bertrand et son équipe, notamment Adham, Xavier et Emeline.
Notre collaboration n'aurait pas été aussi réussie sans votre accueil enthousiaste à Montpellier, ni votre expertise.
Enfin, c'est aux membres de mon jury que je souhaite adresser de sincères remerciements.
Etienne Decencière, Carolina Wählby, Marcelo Nollmann et Perrine Paul-Gilloteaux, merci de me faire l'honneur de lire et de juger mes travaux de thèse.\\

Ces dernières années si particulières ont pu mettre en évidence à quel point la vie d'un laboratoire pouvait être importante pour notre recherche.
Merci à la grande famille du CBIO pour votre enthousiasme, les discussions, les pauses cafés, et tout simplement pour partager les bons et les mauvais moments d'une thèse.

Mes pensées vont tout particulièrement à Chloé, Florian et Véronique pour leur leadership et leur bienveillance au quotidien.
J'aimerais mentionner Asma, avec qui j'ai commencé cette aventure au CBIO, ainsi que la vieille garde composée de Benoit, Judith, Lotfi, Romain, Joseph et surtout Peter, pour votre accueil chaleureux à mon arrivée dans l'équipe.
Merci à Thomas B. et Mélanie, pour cette improbable conférence en Terre Sainte, à Thomas D. qui, plus que tout autre dans l'équipe, a pu partager mes joies et mes peines du FISH et à Maxence, dont je garde un très bon souvenir de notre \emph{paper reading group} du vendredi.
Pouvoir encadrer ton stage a été une très belle expérience, malgré des conditions qui n'ont pas été toujours évidentes.
Je remercie également Gwenaëlle, Julie, Aurélie et Philippe pour m'avoir fait découvrir vos talents de danseur, Matthieu N. (avec qui j'ai l'impression de partager 950 amis et connaissances) pour faire Movember sur 12 mois de l'année, Maguette pour tes précieux conseils en cuisine, Gwenn pour nos débats politiques et discussions hautement philosophiques, Tristan toujours prêt à "jouer" avec le nouveau modèle à la mode, Elise, dont je regrette de n'avoir pas pu assister à ta superbe soutenance, même si j'avais une très bonne raison ce jour là, Marvin, \emph{a hell of a researcher}, Adeline pour ton énergie et tes \emph{good vibes} au quotidien, Daniel R. pour ta motivation et ton assiduité à participer aux book clubs et autres séminaires, Vincent pour ta curiosité, ton enthousiasme et ta patience lorsqu'il s'agit de nous (ré)apprendre la géométrie et Anne qui a décidé de mener de front une carrière de chercheuse et de médecin (parce qu'une seule des deux aurait été trop facile).
La vie d'un laboratoire ne se limite pas à la recherche et c'est pourquoi je remercie grandement Caroline, Katy et Pamela pour leur aide quotidienne aux Mines ou à l'Institut Curie, et surtout pour m'avoir supporté avec ma phobie administrative.\\

Ma dernière salve de remerciements est réservée aux personnes extérieures au CBIO qui m'ont accompagnée, formée et soutenue.

Merci Mr. Peltan et Mr. Brochier, je vous dois beaucoup.
Quand je suis venu vous voir il y a quelques années avec l'idée un peu loufoque de passer un concours en candidat libre, à ma grande surprise, vous avez accepté de m'aider.
Sans vous, mon parcours aurait été bien différent.
Irène, Gaël et Catherine, merci de m'avoir mentoré.
Tous les trois docteurs, tous les trois extrêmement compétents et bienveillants, c'est avec vos conseils et votre exemple en tête que je me suis lancé dans cette thèse.

Un grand merci enfin à mes amis et à ma famille.
Merci à Thibaut toujours présent malgré l'éloignement et le temps qui passe et que je ne saurais classer parmi les amis ou la famille.
Merci à Guillaume, Lucas, Théo, Florian V., Mathieu M., Hugo, Mathieu G., Adrien, Claire, Victor, Florian L. et Alexandre pour tous les weekends, les soirées (oubliées ou à oublier), les brunchs, les vacances et autres souvenirs partagés avec vous.
Merci à la famille du Caire, Elisa et Gauthier, pour avoir rendu la vie parisienne si agréable.
Merci à mes grand-parents.
Néné, comme tu me l'as si bien dit, cette thèse c'est aussi un peu grâce à toi.
Papa et Maman, merci pour votre soutien et votre amour.
Merci Juna, merci Poppy, pour me donner la meilleure des raisons pour descendre les weekends à Marseille.

Finalement, ces remerciements ne seraient pas complets sans un dernier mot pour Mary, celle dont le soutien a été si important pour moi ces dernières années.
L'aventure n'aurait pas été aussi belle sans toi.
Du fond du coeur, merci.