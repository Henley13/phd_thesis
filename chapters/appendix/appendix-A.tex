%!TEX root = ../../main.tex

\graphicspath{{../../figures/appendix/}}

\chapter{Simulation and detection} \label{ch:appendix_simulation_detection}

\newpage

\section{Spots and cluster simulations} \label{sec:appendix_simulations}

\vfill

Figure~\ref{fig:spots_mosaic} is a sample of simulated images of spots.
We modulate the number of spots and the intensity of the noise for each image.

\vfill

\begin{figure}[h]
    \centering
    \includegraphics[width=\textwidth]{figures/appendix/spots_mosaic}
    \caption{Simulations of spots under different noise regimes}
    \label{fig:spots_mosaic}
\end{figure}

\newpage

\null
\vfill

Figure~\ref{fig:cluster_mosaic} use the same logic but with an unique cluster of spots simulated in the center.
This time, we modulate the number of spots inside the cluster.

\vfill

\begin{figure}[h]
    \centering
    \includegraphics[width=\textwidth]{figures/appendix/cluster_mosaic}
    \caption{Simulations of cluster under different noise regimes}
    \label{fig:cluster_mosaic}
\end{figure}

\newpage

\section{Detection with different noise intensities} \label{sec:appendix_detection}

\vfill

In figure~\ref{fig:general_spots_elbow} we present elbow curves for three levels of noise.
We observe the number of detected spots as a function of intensity thresholds.
The optimal threshold (in red) is selected based on these curves.
When an image has a high \ac{SNR}, the difference of regime between the noisy background blobs and the actual spots is clearly distinct in the elbow curve.

The result in term of detection can be observed in figure~\ref{fig:general_spots_detection}.
For each image, 100 spots are simulated.
Detected positions are in red and simulated ground truth positions in white.

\vfill

\begin{figure}[h]
    \centering
    \includegraphics[width=1\textwidth]{figures/appendix/spots_elbow}
    \caption{Elbow curves for different noise levels}
    \label{fig:general_spots_elbow}
\end{figure}

\begin{figure}[h]
    \centering
    \includegraphics[width=1\textwidth]{figures/appendix/spots_example}
    \caption{Detection of 100 spots with different noise levels}
    \label{fig:general_spots_detection}
\end{figure}
