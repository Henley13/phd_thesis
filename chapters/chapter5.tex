%!TEX root = ../main.tex

\graphicspath{{../../figures/chapter_5/}}

\part{Applications}
\chapter{Localizing \ac{mRNA}s} \label{chap:chapter_5}
\minitoc
\newpage


\section{State of the art}


\subsection{Battich}

\subsection{Pichon}

\subsection{Cai}


\section{General pattern recognition}


\subsection{Semi-automated detection (\ac{RNA})}

\subsection{NucleAIzer and watershed}

\subsection{Binary classification models}

\subsection{Gene aggregated results}

\subsection{t-SNE visualization}

3.4 Unsupervised visualization
Another part of our work concerns the representation of thousands of cells
based on localization features. We adopt an unsupervised approach. From the
hand-crafted features or those built automatically by the neural network, we
have a representation of each cell in a multidimensional point cloud. We use
a t-distributed stochastic neighbour embedding (t-SNE) [38] to get a 2-dimensional
projection of such a point cloud. The algorithm computes the probability distribution
relative to the similarities of different observations in our high-dimensional
space and the same probability distribution for observations in a 2-dimensional
space. The 2-dimensional projection which minimizes the KL-divergence between
both distributions is returned. Figure 13 is a 2-dimensional projection of the
high-dimensional vectors returned by the neural network we describe in the
previous section. The point cloud represents 4,000 real cells extracted from
experimental images. Some clusters appear, especially for the foci, nuclear
retention and extension patterns. Examples of relative input images are given
in the figure. Different genes can present the same pattern. On the contrary,
a same gene can present heterogeneous patterns. For example, if most of the
time CEP192 mRNAs are kept inside the nucleus, they are periodically released
in the cytoplasm.

\section{Translation factories}


\subsection{Localized translation}

\subsection{Puromycin impact}


\section{Centrosomal pattern}


\subsection{Fully automated detection (\ac{RNA} and centrosome)}

\subsection{Cellpose}

\subsection{descriptive statistics}

\subsection{Influence of mitosis}


\section{Protrusion pattern}


\subsection{A mature pipeline}

\subsection{Descriptive statistics}

\subsection{Role of KIF1C}


\section{Exploring large scale dataset}
