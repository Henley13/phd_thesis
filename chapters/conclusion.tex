%!TEX root = ../main.tex

\chapter{Conclusion and perspectives}
\label{ch:conclusion}

\minitoc
\newpage

The objective of my PhD thesis was to design methods and software for the large-scale analysis and quantification of \ac{RNA} localization patterns from \ac{smFISH} images and to apply these methods to large-scale screens.
This provides new opportunities to explore more advanced algorithms, incorporate new input modalities, or answer more challenging questions.

\section{Results of the thesis}
\label{sec:conclusion_thesis}

In \textbf{Chapter~\ref{ch:chapter1}}, I present FISH-quant v2, the general framework implemented and released during my PhD~\cite{Imbert_fq_2022}.
It is a user-friendly, open-source and Python-based tool that builds on and extends the initial MATLAB version of FISH-quant providing modules for the most common operations needed for a \ac{smFISH} analysis.
FISH-quant v2 features a fully automatic spot detection and pre-trained deep learning models for cell and nuclei segmentation and a set of hand-crafted features designed to quantitatively describe the spatial distribution of \ac{RNA} inside cells.
The software architecture is highly modular and flexible, allowing the design of customized workflows.
FISH-quant v2 includes \emph{bigfish}, a general Python package for the analysis, a \ac{GUI}, the module \emph{simfish}, a Python package for simulations, as well as documentation and interactive examples.
In addition, the protocol and the analysis pipeline we have performed for different applications are described in a recent publication~\cite{safieddine_ht_smfish_2022}.

In \textbf{Chapter~\ref{ch:chapter2}}, I present detection methods integrated in FISH-quant v2.
An important improvement concerns the spot detection itself with the implementation of a heuristic to threshold the detected spots without any manual intervention.
This enables a large scale application of our methods.
Furthermore, I describe solutions to decompose regions with cluttered spots, detect clusters or analyze spot colocalization.
Finally, I explain how spot images are simulated in FISH-quant v2.
Simulation plays a key role for the performance assessment of spot detection under different noise conditions.

In \textbf{Chapter~\ref{ch:chapter3}}, I present methods integrated in FISH-quant v2 to segment nuclei and cells.
It includes in-house deep learning models trained on various fluorescence microscopy images.
Most importantly, the ability to use external frameworks and models is facilitated with postprocessing methods to format and refine segmentation masks.
In addition, I describe contributions to other projects aiming at overcoming the need for massive image annotation for image segmentation, for instance by the use of in silico labeling for pre-training~\cite{Bonte_2022}.

In \textbf{Chapter~\ref{ch:chapter4}}, I present my different contributions to analyze \ac{RNA} distribution.
First, previous detection and segmentation results are summarized in a homogeneous coordinate representation of each identified cell.
Second, I present a set of features describing the spatial distribution of spots, which I implemented in FISH-quant v2, based on the extracted coordinates.
This feature set is building on and extending features that have been previously used in our research group.
Third, I present alternative approaches to classify \ac{RNA} localization patterns, either with \ac{CNN}~\cite{dubois_deep_2019} or point cloud models~\cite{pointfish_2022}.
These alternatives also leverage a simulation framework that generates localization patterns at will and available in FISH-quant v2.

In \textbf{Chapter~\ref{ch:chapter5}}, I present three (published) applications of my quantitative methods.
In the first study~\cite{CHOUAIB_2020}, I design a classification pipeline to discriminate several generic \ac{RNA} localization patterns: intranuclear, nuclear edge, perinuclear, foci and protrusion.
I also detail additional quantifications performed on the impact of translation inhibitors for a newly characterized pattern, the \emph{translation factories}.
In the second study~\cite{safieddine_choreography_2021}, I focus on the centrosomal pattern and adapt my analysis pipeline to detect centrosomes at scale.
This enables a statistical description of several genes of interest.
In the third study~\cite{pichon_kinesin_2021}, I analyze the protrusion pattern and implement dedicated features recently described in the literature.
All in all, these studies include several dozens of transcripts, analyzed through 100,000 individual cells.
They reveal regulation mechanisms more complex than expected with various spatial and temporal dynamics.

\section{Perspectives on the future of smFISH analysis}
\label{sec:conclusion_future}

% better images --> easier task
Since the first design of \ac{smFISH} in the late 1990s, fluorescent microscopy techniques have dramatically improved.
Fluorophores are cheaper or brighter, or both.
The hybridization protocols are standardized and thus more reproducible.
The fluorescent images have a higher \ac{SNR} and this trend is likely to continue.
Therefore, in future experimental studies, the detection and segmentation steps could be facilitated.

% learning from simulations --> domain adaptation
\paragraph{Improving localization features}
In chapter~\ref{ch:chapter4}, I have presented PointFISH, a neural network architecture that can directly process point clouds.
This network was trained on simulated data.
There is a number of possible improvements: first, we could improve the simulations themselves, for instance by investigating a larger diversity of patterns.
An in-depth study of the relation between pattern strength and generalization capacity of the trained network would also be an interesting perspective.
Furthermore, training on simulated data always represents the problem that the data distribution we have trained on does usually not match the distribution of real data.
For this reason, the use of domain adaptation might be a promising strategy to investigate.
Finally, the use of self-supervised learning is also an interesting alternative to training on simulations.

% more compartments labeled
% in silico labeling
\paragraph{Reference markers}
The analysis of \ac{RNA} localization requires the segmentation of the cell boundary and ideally of structures inside the cell.
The insights that can be gained from \ac{smFISH} studies very much depend on the choice of the marked structures, as we have seen in chapter~\ref{ch:chapter5}.
An important extension is therefore the use of in silico labeling techniques.
This would allow for instance to segment certain structures without the use of fluorescent labels, which in turn frees these channels to mark other structures.
Such a strategy thus has the potential to enrich the description of the spatial \ac{RNA} distribution by adding more reference structures.

% cell stratification
\paragraph{Analysis of localization heterogeneity}
Another question that has not yet been addressed in detail is the explanation of localization heterogeneity.
Indeed, we have seen in chapter~\ref{ch:chapter5}, that for a given \ac{RNA} not all cells show the same localization pattern, and that only a fraction of \ac{RNA} molecules has preferential localization.
It is not well understood where these differences come from.
One possible explanation might be the cell cycle phase.
It will therefore be interesting to assess the cell cycle phase, ideally by in silico labeling from phase contrast images or dedicated fluorescent markers, in order to stratify cells and to investigate which of the patterns are cell cycle dependent.
On a larger perspective, other cell properties could be correlated to the observed patterns.
For this reason, one important aspect to be investigated in the future will be to relate the localization information to other cellular properties.

% more channels
\paragraph{Analyzing transcriptomic profiles}
The most important trend today is the advent of experimental techniques such as MERFISH and seqFISH, that allow the imaging of tens or hundreds of \ac{RNA}s in the same cells.
From a computational point of view, the algorithmic steps (cell and nucleus segmentation, spot detection, etc.) still remain the same, but they have to be complemented by the joint analysis of different \ac{RNA} channels.
In most studies, these data are used to map each cell to a cell type or cell state, that is defined by the transcriptomic profile.
If used in cell culture, this means that each cell is then equipped with an expression vector of multiple \ac{RNA}s, the resulting cell type or state, the localization features for each of these \ac{RNA}s and the morphological properties of the cell.
The biggest challenge in these data is to deal with these high dimensional input data, to identify relationships between the input variables and to synthesize this extremely rich information to representations that can be interpreted biologically.
FISH-quant v2 offers solutions to some of these tasks, by providing tools for the basic image analysis and the analysis of localization features, but it is clear that we will need important extensions to tackle these additional challenges.

% tissue
\paragraph{From cell culture to tissue}
While there are studies employing these transcriptomic techniques in cell culture, the most widely used application scenario concerns the study of tissue architecture.
While in principle, the overall algorithmic steps remain similar, segmentation can be a bottleneck, as it is much more difficult in tissue than in cell culture.
Moreover, the degree of difficulty is very variable depending on the tissue type.
Also spot detection can be complicated by the presence of auto-fluorescence, possibly requiring more refinement steps.
Finally, the main task of such data is also different, because the \ac{RNA} channels are mostly used for cell type calling.
An important challenge is to describe the spatial distribution of cells in the tissue.
For this, there might be some methodological overlap with the analysis of subcellular localization patterns, but there are important differences too.
Most importantly, spatial transcriptomics at cellular resolution is an expensive technique, and purely data-driven approaches might therefore be limited by the low number of data points.

% segmentation
\paragraph{Data repositories and challenges}
On a more general perspective, bioimage analysis will continue to be driven by the generation of massive annotated datasets, and challenges.
In the past, such challenges played an important role for the design and benchmarking of segmentation networks, such as Stardist, NucleAIzer and Cellpose.
For protein localization patterns, the availability of large scale annotated data has also been instrumental.
For these reasons, it would be very interesting to organize a challenge in the field of \ac{RNA} detection and classification of localization patterns.
Our simulation framework could play an important role for this: computational challenges often contain one sub-task related to simulated image data.

Altogether, new experimental methods, accompanied by computational methods, will enable a more efficient exploration of the transcriptome, a better understanding of the mechanism and function of \ac{RNA} localization and the spatial regulation of gene expression.
Obviously, there is still a lot of work to achieve for future researchers and I hope that my contribution with this thesis will be useful to the community.

\section{Publications}
\label{sec:conclusion_publications}

\subsubsection{First author or co-first author}

\begin{itemize}
	\item Racha Chouaib\footnote{Equal contribution}, Adham Safieddine\footnotemark[1], Xavier Pichon\footnotemark[1], \textbf{Arthur Imbert\footnotemark[1]}, Oh Sung Kwon, Aubin Samacoits, Abdel-Meneem Traboulsi, Marie-Cécile Robert, Nikolay Tsanov, Emeline Coleno, Ina Poser, Christophe Zimmer, Anthony Hyman, Hervé Le Hir, Kazem Zibara, Marion Peter, Florian Mueller, Thomas Walter, Edouard Bertrand (2020),\textit{A dual protein-mRNA localization screen reveals compartmentalized translation and widespread co-translational RNA targeting}, Developmental Cell 54 (6), 773.
	\item \textbf{Arthur Imbert}, Wei Ouyang, Adham Safieddine, Emeline Coleno, Christophe Zimmer, Edouard Bertrand, Thomas Walter, Florian Mueller (2022), \textit{FISH-quant v2: a scalable and modular tool for smFISH image analysis}, RNA, pp. $\operatorname{786--795}$, iSSN: $\operatorname{1355--8382, 1469--9001}$.
	\item \textbf{Arthur Imbert}, Florian Mueller, Thomas Walter (2022), \textit{PointFISH: learning point cloud representations for RNA localization patterns}, in 2022 European Conference on Computer Vision (ECCV 2022) Workshop on BioImage Computing \textit{(to be published)}.
\end{itemize}

\subsubsection{Co-author}

\begin{itemize}
	\item Rémy Dubois, \textbf{Arthur Imbert}, Aubin Samacoïts, Marion Peter, Edouard Bertrand, Florian Mueller, Thomas Walter (2019), \textit{A Deep Learning Approach To Identify mRNA Localization Patterns}, in 2019 IEEE 16th International Symposium on Biomedical Imaging (ISBI 2019), pp. $\operatorname{1386-1390}$, iSSN: $\operatorname{1945-8452}$.
	\item Adham Safieddine, Emeline Coleno, Soha Salloum\footnotemark[1], \textbf{Arthur Imbert\footnotemark[1]}, Abdel-Meneem Traboulsi, Oh Sung Kwon, Frederic Lionneton, Virginie Georget, Marie-Cécile Robert, Thierry Gostan, Charles-Henri Lecellier, Racha Chouaib, Xavier Pichon, Hervé Le Hir, Kazem Zibara, Florian Mueller, Thomas Walter, Marion Peter, Edouard Bertrand (2021), \textit{A choreography of centrosomal mRNAs reveals a conserved localization mechanism involving active polysome transport}, Nature Communications 12 (1), 1352.
	\item Xavier Pichon\footnotemark[1], Konstadinos Moissoglu\footnotemark[1], Emeline Coleno, Tianhong Wang, \textbf{Arthur Imbert}, Marie-Cecile Robert, Marion Peter, Racha Chouaib, Thomas Walter, Florian Mueller, Kazem Zibara, Edouard Bertrand, Stavroula Mili (2021), \textit{The kinesin KIF1C transports APC-dependent mRNAs to cell protrusions}, RNA 27 (12), 1528.
	\item Adham Safieddine, Emeline Coleno, Frederic Lionneton, Abdel-Meneem Traboulsi, Soha Salloum, Charles-Henri Lecellier, Thierry Gostan, Virginie Georget, Cédric Hassen-Khodja, \textbf{Arthur Imbert}, Florian Mueller, Thomas Walter, Marion Peter, Edouard Bertrand (2022), \textit{HT-smFISH: a cost-effective and flexible workflow for high-throughput single-molecule RNA imaging}, Nature Protocol.
	\item Thomas Bonte, Maxence Philbert, Emeline Coleno, Edouard Bertrand, \textbf{Arthur Imbert}, Thomas Walter (2022), \textit{Learning with minimal effort: leveraging in silico labeling for cell and nucleus segmentation}, in 2022 European Conference on Computer Vision (ECCV 2022) Workshop on BioImage Computing \textit{(to be published)}.
\end{itemize}