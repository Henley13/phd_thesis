%!TEX root = ../main.tex

\graphicspath{{../../figures/chapter_4/}}


\chapter{Analysis} \label{chap:chapter_4}
\minitoc
\newpage


\section{State of the art}



% ML and DL (pointnet) to classify caveolae clusters and non-caveolae clusters
%~\cite{khater_caveolae_2019}
% Specifically, we are generating abalanced dataset of 1000 blobs of isotropic
% point clouds and 1000 blobs of non-isotropic point clouds. The isotropic
% class of blobs mimicking the caveolae (positive class) and the non-isotropic
% class mimicking the non-caveolae (negative class). The non-isotropic class of
% blobs are more planar structures, while the isotropic class are more spherical
% structures.

% Caveolae are plasma membrane invaginations whose formation requires caveolin-1 (Cav1),
% the adaptor protein polymerase I,and the transcript release factor (PTRF orCAVIN1).
% Caveolae have an important role incell functioning, signaling, and disease. Inthe
% absence ofCAVIN1/PTRF, Cav1 forms non-caveolarmembrane domains called scaffolds.
% Inthis work, we train machine learning models toautomatically distinguish between
% caveolae and scaffolds from single molecule localization microscopy (SMLM) data
% The first uses arandom forest classifier applied to28 hand-crafted/designed features
% (expert features), the second uses aconvolutional neural net (CNN) applied toaprojection
% ofthe point clouds onto three planes, and the third uses aPointNet model, a recent
% development that can directly take point clouds as its input.


\subsection{Classical \ac{FISH} analysis}

\subsection{Spatial statistics}

\subsection{FISH-Quant v1}

\subsection{RDI calculator}

\subsection{DypFISH}


\section{Cell extraction}


\subsection{Results matching and cell extraction}

\subsection{Expression level and quantification and cluster enrichment}


\section{Hand-crafted localization features}


\subsection{Distance features}

\subsection{Dispersion features}

\subsection{Morphological features}

\subsection{Centrosomal features}


\section{Learned features}


\subsection{Pretext task: simulated pattern classification}

\subsection{PointNetMorphology model}

\subsection{Embedding extraction}
