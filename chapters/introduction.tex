%!TEX root = ../main.tex

\graphicspath{{./figures/introduction/}}

\chapter{Image-based transcriptomics}
\label{ch:introduction}

\minitoc
\newpage

\section{Gene expression process}
\label{sec:gene_expression}

\subsection{Transcription}

\subsection{Translation}

\subsection{Transport mechanisms}

\subsection{\ac{mRNA} localization}

% paper racha (introduction)

how
- rna metabolism
- protein metabolism (mature protein or nascent protein)

why
- storage of untranslated rna
- local translation
- rapid cell division cycle (linked to local translation of cyclin B mRNAs)
- cell polarization
- cell motility (actin at the leading edge)
- axonal growth and synaptic plasticity of neurons
- assembly of protein complexes
- avoid proteins in wrong place

how
- targeting signal of nascent protein
- targeting from rna zip-code sequence (with RNA-binding proteins)
- direct transport on the cytoskeleton by molecular motors
- anchoring mechanism
- diffusion, trapping, degradation and local rna stabilization

claim
- lack a global view of local translation in the entire cellular space or at the genomic level
- need a systematic manner

examples
- pioneering study in Drosophila
- human cell lines

limitations
- information on RNA localization but did not directly investigate local translation as the encoded proteins were not detected

% Most mRNAs are distributed randomly throughout the cytoplasm, but some localize to specific
% subcellular areas (Blower, 2013; Bovaird et al., 2018; Eliscovich and Singer, 2017; Jung et al., 2014 for reviews).

% This phenomenon is linked to either RNA metabolism, when untranslated mRNAs are stored in P- bodies or other cellular structures (Hubstenberger et al., 2017),
% or to protein metabolism, when a protein is synthesized locally.

% Local translation has been observed from bacteria and yeast to humans (Blower, 2013; Jung et al., 2014; Eliscovich and Singer, 2017; Bovaird et al., 2018).

% It is commonly involved in the delivery of mature proteins to specific cellular compartments, while allowing local regulation, and this is involved in many processes.
% For instance, it contributes to patterning and cell fate determination during metazoan development, mainly through asymmetric cell division
% (Melton, 1987; Driever and Nu€sslein- Volhard, 1988).
% In Xenopus embryos, local translation of cyclin B mRNAs at the mitotic spindle is also believed to be important for the rapid cell
% division cycles occurring during early embryo- genesis (Groisman et al., 2000).
% In mammals, mRNA localization is involved in cell polarization and motility, mainly through the localization of actin and related mRNAs at the leading edge (Lawrence and Singer, 1986), and it is also involved in axonal
% growth and synaptic plasticity of neurons (Van Driesche and Martin, 2018).

% Importantly, local translation can also be linked to the
% metabolism of the nascent peptide rather than to directly localize the mature protein. For instance, translation of secreted proteins
% at the endoplasmic reticulum (ER) allows nascent pro- teins to translocate through the membrane to reach the ER lumen (Aviram and Schuldiner, 2017).
% Translation of mRNAs at specific sites may also be important for the assembly of protein complexes (Pichon et al., 2016) or
% to avoid the deleterious effects of releasing free proteins at inappropriate places (Mu€ller et al., 2013).

% RNA localization can be accomplished through several mech- anisms. In the case of secreted proteins, the nascent peptide serves as a
% targeting signal, via the signal recognition particle (SRP) and its receptor on the ER (Aviram and Schuldiner, 2017).

% In most other cases, targeting is an RNA-driven process (Blower, 2013; Jung et al., 2014; Eliscovich and Singer, 2017; Bovaird et al., 2018 for reviews).
% Localized mRNAs often contain a zip-code sequence, frequently located within their 30-UTR, which is necessary and sufficient to transport them to their destination.
% The zip code is recognized by one or several RNA-binding proteins (RBPs), and it drives the formation of a transport complex sometimes called locasome.
% This complex can be transported by centrosomes, endosomal vesicles, or other cellular structures (Blower, 2013; Jung et al., 2014; Eliscovich and Singer, 2017; Bovaird et al., 2018).

% However, direct transport on the cytoskeleton by molecular motors is a frequent mechanism (Blower, 2013; Bovaird et al., 2018; Eliscovich and Singer, 2017; Jung et al., 2014).

% Once at destination, an anchoring mechanism may limit diffusion away from the target site.

% Alternatives to these transport mechanisms include diffusion and trap- ping at specific locations and degradation coupled to local RNA stabilization.

% Localized mRNAs are often subjected to a spatial control of translation (Besse and Ephrussi, 2008).
% In the case of Ash1 mRNA in yeast and b-actin mRNA in neurons, translation is repressed during transport and
% is activated at their final location by phosphorylation-dependent mechanisms (Hu€ttelmaier et al., 2005; Paquin et al., 2007).
% This spatial regulation of translation provides an additional layer of control ensuring that mRNAs are translated only at the desired location.

% The first locally translated mRNAs were found by chance or using a candidate approach.
% Purification of cellular structures and localized RBPs have significantly increased the number of known localized mRNAs
% (Blower, 2013; Jung et al., 2014; Eliscovich and Singer, 2017; Bovaird et al., 2018).

% However, only specific compartments or RBPs were examined, and we currently lack a global view of local translation in the entire cellular space or at the genomic level.
% Few reports described attempts to characterize mRNA localization in a systematic manner.

% A pioneering study in Drosophila used whole-mount fluorescent in situ hybrid- ization (FISH) to analyze the localization of more than 2,000 mRNAs (Lecuyer et al., 2007).
% As many as 71\% of them had a non-random distribution, and a range of new localization patterns were observed.
% More recent reports confirmed that RNA localization is widespread during Drosophila development (Jam- bor et al., 2015; Wilk et al., 2016).

% However, it is not known whether this is also true in other organisms, particularly in humans. Few recent studies addressed this question in cell lines
% using the more sensitive single-molecule FISH technique (smFISH). Several thousands of mRNAs were analyzed, which
% showed a correlation of intracellular mRNA distribution with gene annotation (Battich et al., 2013; Chen et al., 2015; Eng et al., 2019; Xia et al., 2019).
% Specifically, these studies identified three groups of localized mRNAs, in the perinuclear area, the mitochondria, and the cell periphery,
% the latter being possibly linked to actin metabolism (Chen et al., 2015).

% These studies provided information on RNA localization but did not directly investigate local translation as the encoded proteins were not detected.
% Thus, we still lack a good understanding of the various functions played by local translation at the cellular level.

% In this study, we developed a smFISH screen to specifically address this issue. Using a set of 523 GFP-tagged cell lines spanning
% a variety of cellular functions and an approach that allows simultaneous visualization of mRNA and proteins,
% we found that local translation occurs at various unanticipated locations. In particular, we discovered specialized
% translation factories, where specific mRNAs are translated. These factories are remarkable in that they provide
% a unique mean to regulate the metabolism of nascent proteins and also create a fine granular compartmentalization of translation.

% paper racha (dual protein-rna localization screens)

% In order to simultaneously visualize mRNAs with their encoded proteins, we based our screen on a library of HeLa cell lines,
% each containing a bacterial artificial chromosome (BAC) stably integrated in their genome (Poser et al., 2008).
% Each BAC con- tains a GFP-tagged gene harboring all its regulatory sequences (promoter, enhancers, introns, and 50 and 30 UTRs; Figure 1A).
% The resulting mRNAs are, thus, identical to the endogenous mol- ecules in terms of sequence and isoform diversity, except for the added tag.
% Previous studies showed that such tagged genes are expressed at near endogenous levels and with the proper spatio- temporal pattern (Poser et al., 2008).
% Since the tagged mRNAs contain all the regulatory sequences, we hypothesized that they would localize like the endogenous ones,
% provided that the tag does not interfere with localization. Using BACs offers two advantages. First, a single smFISH probe set
% against the GFP sequence is sufficient to detect all the studied mRNAs. Sec- ond, using mild hybridization conditions,
% GFP fluorescence can be detected together with the smFISH signal (Fusco et al., 2003), and thus, both the mRNA and the encoded protein can be detected in the same cell.

% paper racha (DYNC1H1 motor protein)

% Molecular motors transport cargos to different cellular destina- tions and accumulate themselves at various cellular locations.
% The third pattern, referred to as ‘‘foci,’’ was observed for the dynein heavy-chain mRNA (DYNC1H1).
% This mRNA localized throughout the cytoplasm but aggregated in some bright structures containing several mRNA molecules (Figure 1C), as reported recently (Pichon et al., 2016).

% paper racha (systematic screeening)

% Overall, accumulation of mRNAs in foci was the most frequent pattern (19 out of the total of 32 localized mRNAs; Tables 1 and S2).
% To determine whether any of these foci were P-bodies, we imaged mRNAs together with a P-body marker. Indeed, foci of
% 15 mRNAs co-localized with P-bodies (Figure S5), while the remaining four were distinct structures (BUB1, DYNC1H1, CTNNB1/b-catenin, and ASPM mRNA; Figure S6A).

% paper racha (locally translated mRNAs)

% Having analyzed the localization patterns by a quantitative and unbiased approach, we then examined the co-localization of
% the mRNAs with their encoded protein. Although accumulation of mRNAs in foci was the most frequent pattern (19/32; see Table 1),
% only two mRNAs displayed a faint but detectable GFP signal in their foci: CRKL and CTNNB1/b-catenin (see Figures 2B and S4 for
% CRKL and Figure S11A for CTNNB1/b-catenin). For the other localization patterns, 9 mRNAs co-localized with their pro- tein
% (2\% of the screened mRNAs; Table 1): AKAP1, AKAP9, AP1S2, ASPM, ATP6A2, FLNA, HMMR, NUMA1, and HSP90B1 (Figure 2).
% In some cases, the mRNA/protein co-local- ization was expected. For instance, both HSP90B1 and ATP6A2 contain a signal peptide
% leading to translation on the ER, and HSP90B1 is a resident ER protein while ATP6A2 localizes in endo-lysosomes close to this
% compartment (Figure 2B). Similarly, AKAP1 encodes an RNA-binding protein localizing to the surface of mitochondria, and
% this transcript thus belongs to the known class of mitochondrion-localized mRNAs (Sylvestre et al., 2003). The other proteins
% localized to cellular structures that were not previously known to use local translation as a tar- geting mechanism.
% This included the clathrin adaptor AP1S2 mRNA that localized on endosomes and the AKAP9 mRNA that accumulated at the Golgi.
% Below, we explore these cases in more detail together with mRNAs localizing around centro- somes (ASPM, NUMA1, and HMMR),
% at the nuclear envelope (ASPM and SPEN), or in cytoplasmic foci that were not P-bodies (ASPM, BUB1, DYNC1H1, and CTNNB1/b-catenin).

% racha paper (nuclear envelope rna)

% The Localization of ASPM and SPEN mRNAs at the Nuclear Envelope Is Translation-Dependent
% ASPM-GFP mRNAs accumulated at the nuclear envelope during interphase (Figure 2C; quantification in Figure 3E). In addition,
% SPEN-GFP mRNAs, which encode a nuclear protein, were also enriched around the nuclear envelope (Figures 2B and 3E).
% Labeling of the nuclear pores using either a CRM1-GFP fusion or an antibody against NUP133 (Figures S9A and S9B) showed
% that ASPM and SPEN mRNAs localized close to nuclear pores, rather than between them. This localization could result from
% two mechanisms. First, mRNAs could transiently localize at the pores on their way out of the nucleus. Second, they could
% localize at the cytoplasmic side of the pore to facilitate re-entry of newly translated proteins. To distinguish between
% these pos- sibilities, we inhibited either transcription with actinomycin D, or translation with puromycin. After 1 h of
% actinomycin D treatment, SPEN and ASPM mRNAs still localized at the nuclear envelope (left panels in Figures 4A, S9C, and
% 4B for quantifications). On the contrary, both mRNAs became dispersed after 1-h exposure to puromycin (Figures 4A, 4B, S9C,
% and S10A for quantifica- tions). We then used cycloheximide, which inhibits translation by freezing the ribosomes on the mRNAs,
% in contrast to puromy- cin that induces premature termination and releases the nascent peptide. Cycloheximide had no effects
% on the localization of ASPM mRNAs (Figures 4A left panels and 4B), indicating that their localization did not require protein
% synthesis per se, but more likely the presence of nascent proteins on polysomes. Therefore, ASPM and SPEN mRNAs at the nuclear
% envelope are not in transit to the cytoplasm but localize on the nuclear en- velope by a translation-dependent mechanism requiring the nascent protein.

% ASPM mRNAs Are Translated at the Nuclear Envelope
% We and others recently demonstrated that the SunTag can be used to visualize translation of single mRNPs in live cells
% (Pichon et al., 2016; Wu et al., 2016; Yan et al., 2016; reviewed in Pichon et al., 2018). The SunTag is a repeated
% epitope and in cells ex- pressing an scFV-GFP monochain antibody directed against this epitope, the antibody binds
% the repeated tag as soon as it emerges from the ribosome, allowing the visualization of single molecules of nascent proteins
% (e.g., monosome and polysomes). To confirm that ASPM mRNAs were translated at the nuclear en- velope, we tagged the
% endogenous gene using CRISPR genome editing. Homologous recombination allowed to obtain heterozy- gous clones where
% 32 SunTag repeats were fused at the N termi- nus of ASPM proteins (Figure S9D). The endogenous ASPM tran- scripts
% were then labeled by smiFISH, revealing both tagged and untagged mRNAs. This showed that bright spots of scFv-GFP
% co-localized with single ASPM mRNAs (Figure 4C; white arrows). These spots disappeared after a 20 min puromycin
% treatment, confirming that they were polysomes (Figure S9E). We then per- formed time-lapse analyses of live cells,
% recording one image stack every 40 s for 50 min (see Video S1 and still images in Fig- ure 4D). Remarkably,
% some polysomes localized at the nuclear envelope. While cytoplasmic polysomes moved too rapidly to be tracked, the ones at the envelope remained immobile for
% extended periods of times (29 min on average). Thus, a fraction of ASPM mRNAs was stably anchored and translated at the nu- clear envelope.

% racha paper (use of puromycin)

% Translation Inhibition with Puromycin Frequently Prevents mRNA Localization
% Since the localization of ASPM mRNA at the envelope is transla- tion dependent, we tested whether this was also the case for its
% localization at centrosomes. Puromycin abolished its localiza- tion, while actinomycin D and cycloheximide had little effect
% (Fig- ure 4A, right panels; Figure 4B for quantification). The effect of puromycin was then tested on the other localized mRNAs.
% After 1 h of treatment, KIF1C and MYH3 mRNAs still localized to cyto- plasmic protrusions (Figures 5A and 5B for quantifications).
% In contrast, HSP90B1, HMMR, AP1S2, AKAP1, KIF4A, and AKAP9 mRNAs all became delocalized and lost co-localization with their
% ncoded protein when translation was inhibited (Fig- ures 5C–5E and S10A–S10D). The data demonstrated are consistent with the
% hypothesis that translation is required for the localization of these mRNAs.

% racha paper (translation factories)

% Non-P-Body mRNA Foci Correspond to Specialized Translation Factories
% Four mRNAs accumulated in foci that were distinct from P- bodies (BUB1, DYNC1H1, CTNNB1/b-catenin, and ASPM; Fig- ure S6A).
% The dynein heavy-chain DYNC1H1 and the ASPM pro- tein were described above. BUB1 is a checkpoint kinase that
% verifies the attachment of microtubules to kinetochores at the onset of mitosis (Saurin, 2018), and b-catenin is the
% key tran- scription factor of the Wnt signaling pathway (Grainger and Wil- lert, 2018). SmiFISH confirmed that the four
% endogenous mRNAs accumulated in foci (Figures S2A, S8C, S10E, and see below S11F). Moreover, dual-color smiFISH showed
% that these foci did not co-localize together, indicating that they were distinct structures (Figure 6A).
% We then inhibited translation with puro- mycin, using an mRNA accumulating in P-bodies as control (AURKA).
% After 1 h of treatment, the non-P-body foci virtually dis- appeared (Figures 6B and 6C), while the accumulation
% of AURKA mRNA in P-bodies actually increased. Thus, translation inhibition specifically disrupted the mRNA foci
% that were not P-bodies. We previously used the SunTag to show that DYNC1H1 is translated in the mRNA foci (Pichon et al., 2016).
% Using the CRISPR SunTag clone described above, we found that ASPM mRNAs were simi- larly translated in the mRNA foci
% (Figure 4C, orange arrow; 70\% of the foci have a SunTag signal). A 32xSunTag repeat was then introduced at the N terminus of the BUB1
% protein using CRISPR gene editing. BUB1 mRNAs were detected by smiFISH and were found to be frequently translated in the foci
% (Figures 6D and 6E; 70\% of the foci contained a SunTag signal), in contrast to mRNAs located outside foci for which the SunTag signal was
% rarely detected (1\% of the time). Taken together, these data demonstrated that ASPM, BUB1, and DYNC1H1 are translated
% in mRNA foci, which thus correspond to specialized translation factories.

% racha paper (CTNNB1)

% We then analyzed in more details CTNNB1/b-catenin. This pro- tein has two roles: it bridges E-cadherin to the actin cytoskeleton
% at adherens junctions, and it acts as the main transcription factor of the Wnt pathway (for review, see Grainger and Willert, 2018; MacDonald et al., 2009).
% The Wnt signaling pathway is essential during development, is often a key actor during tumorigenesis, and involves a fast
% activation of b-catenin expression by Wnt. The b-catenin protein localized at adherens junction is stable whereas the one
% synthesized in the cytoplasm is rapidly degraded (reviewed in Stamos and Weis, 2013). However, the cytoplasmic protein is
% stabilized in presence of a Wnt signal and can then accumulate in the nucleus to activate transcription of target genes.
% In the b-catenin BAC cells, the GFP-tagged pro- tein was weakly expressed. It accumulated at sites of cell-cell contacts
% as expected and also showed a weak staining in the brightest RNA foci (Figure S11A). Because the maturation of the GFP
% chromophore is slow and therefore inadequate to map translation sites, we combined smFISH with immunofluo- rescence,
% using an antibody binding the N terminus of b-catenin. The mRNA foci contained high levels of b-catenin N termini (Fig- ure S11B).
% Since these foci were also dissolved when translation was inhibited (see Figures 6B and 6C), these data indicated that b-catenin was translated in the mRNA foci.
% Interestingly, there was a small fraction of the BAC cells where b-catenin mRNAs did not form foci and instead were dispersed as single molecules in the cytoplasm.
% Furthermore, these cells ex- pressed high levels of nuclear b-catenin-GFP, suggesting that the Wnt pathway had been activated and
% was responsible for the disappearance of the foci (Figure S11C). To test this hypothe- sis, we activated the Wnt pathway by
% incubating the BAC cell line with the WNT3A protein. Indeed, the mRNA foci disappeared after 30 min, concomitant with a
% higher expression of b-catenin-GFP and its accumulation in the nucleus (Figures 7A and 7E for quan- tifications).
% This observation established a link between the pres- ence of mRNA foci and the degradation of the b-catenin protein.
% One hypothesis to explain these results would be that b-catenin
% degradation takes place co-translationally in the foci. Degradation of b-catenin requires Axin, APC, and the kinases
% CK1a and GSK3, which altogether form the ‘‘destruction complex’’ (Stamos and Weis, 2013).
% In absence of Wnt, this complex binds b-catenin and targets it for degradation. In presence of Wnt, its components become recruited to the
% plasma membrane, and they can no longer interact with b-catenin (Grainger and Willert, 2018; Stamos and Weis, 2013; MacDonald et al., 2009).
% Axin is an essential player of the destruction complex, and it acts as a scaffolding pro- tein
% (Grainger and Willert, 2018; Stamos and Weis, 2013; Mac- Donald et al., 2009).
% It binds b-catenin as well as APC, CK1a, and GSK3, and its role is to bring b-catenin in proximity to the ki- nases CK1a and GSK3 (3–5).
% b-catenin is first phosphorylated on residue S45 by CK1a, leading to its recognition by GSK3, which phosphorylates
% it on residues T41, S37, and S33. Phosphorylated b-catenin is then recognized by the E3 ubiquitin ligase b-TrCP,
% which targets it for proteasomal degradation. APC is a very large protein that is essential for b-catenin degradation
% but whose exact molecular function is unclear. It occurs as a dimer and each mono- mer has more than 30 tandemly
% repeated binding sites for b-catenin.
% Labeling with anti-APC or anti-Axin1 antibodies revealed that the b-catenin mRNA foci contained high levels of APC
% and were also enriched for Axin1 (Figure 7B). Moreover, Axin1 could be detected in polysomal fractions in a
% translation-dependent manner (Figure S11D), suggesting that it binds b-catenin co- translationally.
% We then immuno-localized phosphorylated b-catenin (on residues S33 and S37), to test its presence in the mRNA foci.
% In this case, we used N-terminal labeling of b-catenin as a proxy for the mRNA foci because the anti-phospho anti- bodies
% were not compatible with smFISH (Figure 7C). Indeed, phospho-b-catenin accumulated in the mRNA foci.
% In addition, we could also detect cross-linked chains of ubiquitin itself (Fig- ure 7C).
% Altogether these data indicated that the mRNA foci were sites of both b-catenin synthesis and degradation, i.e., sites of co-translational protein degradation.
% Foci formation was further regulated by Wnt signaling, thereby allowing a fast post-translational response.
% Interestingly, Axin can oligomerize and APC can cross-link these oligomers to make large aggregates (Grainger and Willert, 2018; Stamos and Weis, 2013; MacDonald et al., 2009),
% suggesting that these factors could be involved in foci formation. Indeed, when APC was knocked down with siRNA,
% we observed a 50-fold increase of b-catenin levels and, remarkably, a disap- pearance of the mRNA foci (Figures 7D and 7F).
% Likewise, mRNA foci were disrupted upon Axin1 knockdown. In contrast, when we increased Axin1 levels by inhibiting
% Tankyrase with the small molecule LG-007 (Huang et al., 2009), the endogenous b-catenin mRNA formed more foci in HEK293 cells:
% foci were visible in 16\% of the cases in untreated cells, and this increased to 32\% after LG-007 treatment for 2 h (Figures S11F and S11G).
% Finally, we also tested a pharmacological inhibition of GSK3. Indeed, APC and Axin1 are themselves phosphorylated by GSK3, and this
% promotes complex formation and their interac- tion with b-catenin (Ha et al., 2004; Wu and Pan, 2010). Remark- ably, GSK3 inhibition
% for 1 h nearly completely dissolved b-cate- nin mRNA foci. These data suggest that foci formation relies on the co-translational
% recognition of b-catenin by APC and Axin1, and the multimerization of these factors.

% other

% https://www.sciencedirect.com/science/article/pii/S2589004221012670

\subsection{Protein localization}

\section{Imaging cells and \ac{RNA}s}
\label{sec:fish}

% paper racha (dual protein-rna localization screens)

% In order to simultaneously visualize mRNAs with their encoded proteins, we based our screen on a library of HeLa cell lines,
% each containing a bacterial artificial chromosome (BAC) stably integrated in their genome (Poser et al., 2008).
% Each BAC con- tains a GFP-tagged gene harboring all its regulatory sequences (promoter, enhancers, introns, and 50 and 30 UTRs; Figure 1A).
% The resulting mRNAs are, thus, identical to the endogenous mol- ecules in terms of sequence and isoform diversity, except for the added tag.
% Previous studies showed that such tagged genes are expressed at near endogenous levels and with the proper spatio- temporal pattern (Poser et al., 2008).
% Since the tagged mRNAs contain all the regulatory sequences, we hypothesized that they would localize like the endogenous ones,
% provided that the tag does not interfere with localization. Using BACs offers two advantages. First, a single smFISH probe set
% against the GFP sequence is sufficient to detect all the studied mRNAs. Sec- ond, using mild hybridization conditions,
% GFP fluorescence can be detected together with the smFISH signal (Fusco et al., 2003), and thus, both the mRNA and the encoded protein can be detected in the same cell.

\subsection{Fluorescent stains and \ac{GFP}}

\subsection{\ac{FISH}}

\subsection{\ac{smiFISH}}

\subsection{\ac{SeqFISH} and \ac{MERFISH}}

\section{Measuring images: from pixels to numbers}
\label{sec:computation_biology}

\subsection{MATLAB, java and python: a zoo of solutions}

\subsection{Detect, segment and analyze}

% unknown
~\cite{shariff_automated_2010}
~\cite{laux_interactive_2020}
~\cite{das_intracellular_2021}

% general pipeline
~\cite{mcquin_cellprofiler_2018}
~\cite{mueller_fish-quant_2013}
~\cite{de_chaumont_icy_2012}
~\cite{ershov_bringing_2021} % trackmate
~\cite{ljosa_introduction_2009}
~\cite{stoeger_computer_2015}
~\cite{perkel_starfish_2019}
~\cite{noauthor_mammalian_2020}
~\cite{eng_transcriptome-scale_2019}
~\cite{kamenova_co-translational_2019}
~\cite{liao_rna_2019}
~\cite{xia_spatial_2019}
~\cite{tsanov_smifish_2016}
~\cite{samacoits_computational_2018}
~\cite{battich_image-based_2013}
~\cite{savulescu_interrogating_2021}

~\cite{battich_image-based_2013}
We obtained 18 primary spot features that reflect the
relative localization of each spot in a single cell, with respect to both the cell and other spots
here we show that branched dnA technology combined with automated liquid handling, high-content imaging and quantitative image analysis allows highly reproducible quantification of transcript abundance in thousands of single cells at single-molecule resolution.
in addition, it allows extraction of a multivariate feature set quantifying subcellular patterning and spatial properties of transcripts and their cell-to-cell variability. this has multiple implications for the functional interpretation of cell-to-
cell variability in gene expression and enables the unbiased identification of functionally relevant in situ signatures
of the transcriptome without the need for perturbations. Because this method can be incorporated in a wide variety of high-throughput image-based approaches, we expect it to be broadly applicable.

~\cite{stoeger_computer_2015}
(talking about ~\cite{battich_image-based_2013})
Therefore, we have previously developed and documented [4] an unsupervised
clustering scheme that uses selected cellular statistics to identify a small
number of main patterns in single cell subcellular transcript localization.
Briefly, this package uses the per-cell mean and standard deviation of the
single-transcript localization features to first identify a number of
different patterns, by clustering random subsets of cells, such that
the clusters are most reproducible. In a second step, it determines the
similarity of each single cell to each of the identified patterns.

To enable imagebased transcriptomics to reach its full potential, we developed
computer vision algorithms that build on and improve those currently used to
detect objects in confocal images. By using iterative watershedding we have
improved the segmentations of nuclei and cells. In addition, we describe how
to perform spot detection for transcript identification in an automated way for
thousands of images. Accurate detection of nuclear outlines, cell outlines, and
transcript molecules are essential for the correct quantification of a
high-dimensional multivariate feature space of each transcript and to reveal
bona fide novel properties of the spatial organization of the transcriptome [4].
The computer vision pipeline presented here complements our earlier work [4],
and can be used independently of transcripts in other image-based approaches.

~\cite{battich_control_2015}
Here, we applied image-based transcriptomics, a highthroughput automated
single-molecule fluorescence in situ hybridization (sm-FISH) method that we
recently developed (Battich et al., 2013), which meets these requirements.
Using large-scale single-cell datasets acquired with this approach, we show
that cell-to-cell variability in cytoplasmic transcript abundance in human
adherent cells can be accurately predicted at the single-cell level with a
multivariate set of features that quantify properties of the cellular state
and microenvironment, and we experimentally verify some of the underlying
causality. We find that for most genes, the unexplained variability in cytoplasmic
transcript abundance approaches a limit of minimal stochasticity imposed by a
Poisson process. The few genes that deviate from this limit also show a high
amount of explained variability, suggesting high-level regulation rather than
high stochasticity. Through computational multiplexing, we uncover the existence
of multilevel transcript homeostasis in single cells to achieve specific
adaptation of transcript abundance to the cellular state and microenvironment,
according to function of the proteins they encode. Finally, we show that the
mammalian nucleus acts as a potent and global buffer to stochastic fluctuations
arising from bursts in gene transcription by temporally retaining transcripts
in the nucleus. This explains how cytoplasmic transcript abundance in mammalian
cells can be minimally stochastic, while deterministic variation is maintained.

% ML for detection ?
~\cite{khater_caveolae_2019}

\subsection{StarFISH: a common \ac{FISH} platform}

\subsection{FISH-Quant MATLAB}

\section{Contributions}
\label{sec:contributions}

\subsection{FISH-Quant v2}

\subsection{Quantifications of localization patterns}
