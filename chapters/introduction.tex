%!TEX root = ../main.tex

\graphicspath{{./figures/introduction/}}

\chapter{Image-based transcriptomics}
\label{ch:introduction}

\minitoc
\newpage

\section{Gene expression process}
\label{sec:gene_expression}

\subsection{Transcription}

\subsection{Translation}

\subsection{Transport mechanisms}

\subsection{\ac{mRNA} localization}

% https://www.sciencedirect.com/science/article/pii/S2589004221012670

\subsection{Protein localization}

\section{Imaging cells and \ac{RNA}s}
\label{sec:fish}

\subsection{Fluorescent stains and \ac{GFP}}

\subsection{\ac{FISH}}

\subsection{\ac{smiFISH}}

\subsection{\ac{SeqFISH} and \ac{MERFISH}}

\section{Measuring images: from pixels to numbers}
\label{sec:computation_biology}

\subsection{MATLAB, java and python: a zoo of solutions}

\subsection{Detect, segment and analyze}

% unknown
~\cite{shariff_automated_2010}
~\cite{laux_interactive_2020}
~\cite{das_intracellular_2021}

% general pipeline
~\cite{mcquin_cellprofiler_2018}
~\cite{mueller_fish-quant_2013}
~\cite{de_chaumont_icy_2012}
~\cite{ershov_bringing_2021} % trackmate
~\cite{ljosa_introduction_2009}
~\cite{stoeger_computer_2015}
~\cite{perkel_starfish_2019}
~\cite{noauthor_mammalian_2020}
~\cite{eng_transcriptome-scale_2019}
~\cite{kamenova_co-translational_2019}
~\cite{liao_rna_2019}
~\cite{xia_spatial_2019}
~\cite{tsanov_smifish_2016}
~\cite{samacoits_computational_2018}
~\cite{battich_image-based_2013}
~\cite{savulescu_interrogating_2021}

~\cite{battich_image-based_2013}
We obtained 18 primary spot features that reflect the
relative localization of each spot in a single cell, with respect to both the cell and other spots
here we show that branched dnA technology combined with automated liquid handling, high-content imaging and quantitative image analysis allows highly reproducible quantification of transcript abundance in thousands of single cells at single-molecule resolution.
in addition, it allows extraction of a multivariate feature set quantifying subcellular patterning and spatial properties of transcripts and their cell-to-cell variability. this has multiple implications for the functional interpretation of cell-to-
cell variability in gene expression and enables the unbiased identification of functionally relevant in situ signatures
of the transcriptome without the need for perturbations. Because this method can be incorporated in a wide variety of high-throughput image-based approaches, we expect it to be broadly applicable.

~\cite{stoeger_computer_2015}
(talking about ~\cite{battich_image-based_2013})
Therefore, we have previously developed and documented [4] an unsupervised
clustering scheme that uses selected cellular statistics to identify a small
number of main patterns in single cell subcellular transcript localization.
Briefly, this package uses the per-cell mean and standard deviation of the
single-transcript localization features to first identify a number of
different patterns, by clustering random subsets of cells, such that
the clusters are most reproducible. In a second step, it determines the
similarity of each single cell to each of the identified patterns.

To enable imagebased transcriptomics to reach its full potential, we developed
computer vision algorithms that build on and improve those currently used to
detect objects in confocal images. By using iterative watershedding we have
improved the segmentations of nuclei and cells. In addition, we describe how
to perform spot detection for transcript identification in an automated way for
thousands of images. Accurate detection of nuclear outlines, cell outlines, and
transcript molecules are essential for the correct quantification of a
high-dimensional multivariate feature space of each transcript and to reveal
bona fide novel properties of the spatial organization of the transcriptome [4].
The computer vision pipeline presented here complements our earlier work [4],
and can be used independently of transcripts in other image-based approaches.

~\cite{battich_control_2015}
Here, we applied image-based transcriptomics, a highthroughput automated
single-molecule fluorescence in situ hybridization (sm-FISH) method that we
recently developed (Battich et al., 2013), which meets these requirements.
Using large-scale single-cell datasets acquired with this approach, we show
that cell-to-cell variability in cytoplasmic transcript abundance in human
adherent cells can be accurately predicted at the single-cell level with a
multivariate set of features that quantify properties of the cellular state
and microenvironment, and we experimentally verify some of the underlying
causality. We find that for most genes, the unexplained variability in cytoplasmic
transcript abundance approaches a limit of minimal stochasticity imposed by a
Poisson process. The few genes that deviate from this limit also show a high
amount of explained variability, suggesting high-level regulation rather than
high stochasticity. Through computational multiplexing, we uncover the existence
of multilevel transcript homeostasis in single cells to achieve specific
adaptation of transcript abundance to the cellular state and microenvironment,
according to function of the proteins they encode. Finally, we show that the
mammalian nucleus acts as a potent and global buffer to stochastic fluctuations
arising from bursts in gene transcription by temporally retaining transcripts
in the nucleus. This explains how cytoplasmic transcript abundance in mammalian
cells can be minimally stochastic, while deterministic variation is maintained.

% ML for detection ?
~\cite{khater_caveolae_2019}

\subsection{StarFISH: a common \ac{FISH} platform}

\subsection{FISH-Quant MATLAB}

\section{Contributions}
\label{sec:contributions}

\subsection{FISH-Quant v2}

\subsection{Quantifications of localization patterns}
