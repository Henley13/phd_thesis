%!TEX root = ../main.tex

\textbf{Abstract:}
\hspace{0.5cm}

\textit{
Each identified element in a cell is represented by its spatial coordinates, resulting in a coordinate representation of the cell.
Two approaches are then presented to study RNA localization patterns.
The first methods is to design hand-crafted features to characterize specific patterns.
These features are implemented in FISH-quant.
In the second approach, a vector representation of the RNA point cloud is learned and extracted from a neural network trained on a simulated task.
Both techniques allow for downstream analyses such as quantification of the RNA distribution, supervised and unsupervised analysis.
}

\vspace{0.5cm}

\noindent
\textbf{Résumé:}
\hspace{0.5cm}

\textit{
Chaque élément identifié dans une cellule est représenté par ses coordonnées spatiales, ce qui permet de représenter la cellule comme un nuage de points.
Deux approches sont ensuite présentées pour étudier les schémas de localisation de l'ARN.
La première méthode consiste à développer manuellement des indicateur statistiques pour caractériser ces schémas.
Ces indicateurs sont implémentés dans FISH-quant.
Dans la seconde approche, une représentation vectorielle du nuage de points d'ARN est apprise et extraite à partir d'un réseau de neurones entraîné sur des données simulées.
Les deux techniques permettent des analyses en aval telles que la quantification de la distribution de l'ARN, l'analyse supervisée et non supervisée.
}