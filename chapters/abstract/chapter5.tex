%!TEX root = ../main.tex

\textbf{Abstract:}
\hspace{0.5cm}

\textit{
FISH-quant was applied in three high content screening studies to extract quantitative evidence from smFISH images.
In the first application, a classification pipeline is developed to identify five different patterns of RNA localization: intranuclear, nuclear edge, perinuclear, foci, and protrusion.
In addition, a novel mechanism of spatial control of gene expression, called translation factories, is characterized.
The second application focuses on the centrosomal pattern.
A modified analysis pipeline is used to detect centrosomes and study the spatial and temporal dynamics of several genes.
The third application analyzes the protrusion pattern, implementing dedicated features to study the role of the motor protein KIF1C in the transport of some APC-dependent transcripts to cell extensions.
In total, these studies involve several dozen transcripts, analyzed through 100,000 individual cells.
}

\vspace{0.5cm}

\noindent
\textbf{Résumé:}
\hspace{0.5cm}

\textit{
FISH-quant a été appliqué dans trois études de criblage à haut débit pour obtenir des analyses quantitatives à partir d'images smFISH.
Dans la première étude, un pipeline de classification supervisée est développé pour identifier cinq schéma de localisation de l'ARN : intranucléaire, membrane nucléaire, périnucléaire, foci et protrusion.
En outre, un nouveau mécanisme de contrôle spatial de l'expression des gènes, appelé usines de traduction, est caractérisé.
La deuxième étude se concentre sur le schéma centrosomal.
Un pipeline d'analyse modifié est utilisé pour détecter les centrosomes et étudier la dynamique spatiale et temporelle de plusieurs gènes.
La troisième étude analyse le schéma de protrusion, en développant des indicateur dédiées pour étudier le rôle de la protéine motrice KIF1C dans le transport de certains ARNs vers les protrusions cellulaires.
Au total, ces études portent sur plusieurs dizaines de gènes, analysés sur 100000 cellules individuelles.
}