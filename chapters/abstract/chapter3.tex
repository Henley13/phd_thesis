%!TEX root = ../main.tex

\textbf{Abstract:}
\hspace{0.5cm}

\textit{
Current solutions for nucleus and cell segmentation are mainly based on deep learning models.
FISH-quant includes such pre-trained models, in addition to postprocessing algorithms to format segmentation masks and refine them.
It also allows the integration of external and potentially more specialized resources.
Two main limitations are then discussed.
First, the need for a large training dataset, which can be mitigated by the use of in-silico techniques.
Second, the lake of consistency between nucleus and cell segmentation.
This consistency can be enforced by the way an algorithm operates, as in the case of a watershed transformation, or learned through a specific training strategy.
}

\vspace{0.5cm}

\noindent
\textbf{Résumé:}
\hspace{0.5cm}

\textit{
Les solutions actuelles pour la segmentation des noyaux et des cellules sont principalement basées sur des modèles d'apprentissage profond.
FISH-quant inclut de tels modèles pré-entraînés, en plus d'algorithmes de post-traitement pour formater les masques de segmentation et les affiner.
Il permet également l'intégration de ressources externes et potentiellement plus spécialisées.
Deux limites principales sont ensuite discutées.
Premièrement, la nécessité d'un grand ensemble de données d'entraînement, limite qui peut être atténuée par l'utilisation de techniques in-silico.
Deuxièmement, le manque de cohérence entre la segmentation des noyaux et des cellules.
Cette cohérence peut être imposée par le mode de fonctionnement d'un algorithme, comme dans le cas d'une segmentation par ligne de partage des eaux, ou apprise par le biais d'une stratégie d'entraînement spécifique.
}
