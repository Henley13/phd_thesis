%!TEX root = ../main.tex

\graphicspath{{../../figures/chapter_3/}}


\chapter{Segmentation} \label{ch:chapter3}
\minitoc
\newpage


\section{State of the art}


\subsection{Watershed and thresholding}

\subsection{Nucleaizer}

\subsection{Stardist}

\subsection{Cellpose}

\subsection{Embedseg}


\section{A variant of deep watershed}


\subsection{Deep watershed and distance prediction}

\subsection{Nucleus-cell joint segmentation}


\section{Benchmark on \ac{FISH} images}


\subsection{A 4-channel dataset}

\subsection{Benchmark}


\section{Postprocessing}

% bigfish.segmentation.remove_segmented_nuc

First, I dilate the binary mask of the segmented nuclei.
Second, in the original DAPI image, every pixels outside of the dilated mask are set to zero.
This includes the background and the potentially missed nuclei.
Third, I perform a morphological reconstruction of the missing nuclei by small dilation.
This dilation is constrained by the original DAPI image.
A pixel can't have a dilated value greater than its original intensity.
This way, the background pixels keep a low intensity and the missed nuclei (brighter in the original image) are partially reconstructed by the dilation.
The reconstructed image only differs from the original one where the nuclei have been missed.
Fourth, I subtract the reconstructed image from the original one to get an approximate image of the missing nuclei.
The latter is used to threshold a binary mask of the missing nuclei and ultimately extract their original pixel intensity from the original image.
Lastly, I merge the masks obtained from the two rounds of segmentation.

% related work (nucleaizer, mask rcnn, unet, single and two pass, coordinates and mask, cellpose, embedseg, Tissuenet)
% related work (watershed, watershed-like model)
% limitation of cellpose
% our small model (single or double input) + fail attempt of deep snake -> consistency nuc/cell
% scale-dependency (peter work ?)
% in silico labelling
% postprocessing (bigfish)