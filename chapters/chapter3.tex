%!TEX root = ../main.tex

\graphicspath{{./figures/chapter3/}}


\chapter{Single-cell segmentation}
\label{ch:chapter3}

\minitoc
\newpage

\begin{center}
	\textit{(To be completed)}
\end{center}

\section{Segmentation of fluorescence microscopy}
\label{sec:segmentation_introduction}

\subsection{Instance segmentation}
\label{subsec:segmentation_instance_introduction}

\begin{center}
	\textit{(To be completed)}
\end{center}

\subsection{Related work}
\label{subsec:segmentation_related_work}

\begin{center}
	\textit{(To be completed)}
\end{center}

% tresholding, watershed, superpixel,
% Unet
% deep watershed, Peter model
% single and two pass
% Nucleaizer, maskRCNN
% Stardist, Cellpose+limitations, densepose?
% embedseg, equivalent model to natural image
% TissueNet

\subsubsection{From mathematical morphology\dots}

\subsubsection{\dots to deep learning models}

% adaptated from natural image model
% importance of dataset
% better performance

\section{Nucleus and cell segmentation}
\label{sec:segmentation_nuc_cell}

\begin{center}
	\textit{(To be completed)}
\end{center}

\subsection{A new multichannel dataset}
\label{subsec:segmentation_data}

% data description

\subsection{Nucleus segmentation}
\label{subsec:segmentation_nuc}

% tresholding
% deep learning model + results
% removing segmented nuclei

% bigfish.segmentation.remove_segmented_nuc

First, I dilate the binary mask of the segmented nuclei.
Second, in the original DAPI image, every pixels outside of the dilated mask are set to zero.
This includes the background and the potentially missed nuclei.
Third, I perform a morphological reconstruction of the missing nuclei by small dilation.
This dilation is constrained by the original DAPI image.
A pixel can't have a dilated value greater than its original intensity.
This way, the background pixels keep a low intensity and the missed nuclei (brighter in the original image) are partially reconstructed by the dilation.
The reconstructed image only differs from the original one where the nuclei have been missed.
Fourth, I subtract the reconstructed image from the original one to get an approximate image of the missing nuclei.
The latter is used to threshold a binary mask of the missing nuclei and ultimately extract their original pixel intensity from the original image.
Lastly, I merge the masks obtained from the two rounds of segmentation.

\subsection{Cell segmentation}
\label{subsec:segmentation_cell}

% watershed
% deep learning model + results
% clean segmentation

\section{Improving cell segmentation}
\label{sec:segmentation_improvements}

\begin{center}
	\textit{(To be completed)}
\end{center}

\subsection{Snake-like model}
\label{subsec:segmentation_snake}

% deep snake + appendix

\subsection{In silico pre-training}
\label{subsec:segmentation_insilico}

% in silico labelling

\section{Conclusion}
\label{sec:segmentation_conclusion}

\begin{center}
	\textit{(To be completed)}
\end{center}